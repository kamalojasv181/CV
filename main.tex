
% !TEX program = xelatex
%%%%%%%%%%%%%%%%%%%%%%%%%%%%%%%%%%%%%%%%%
% Medium Length Professional CV
% LaTeX Template
% Version 2.0 (8/5/13)
%
% This template has been downloaded from:
% http://www.LaTeXTemplates.com
%
% Original author:
% Trey Hunner (http://www.treyhunner.com/)
%
% Important note:
% This template requires the resume.cls file to be in the same directory as the
% .tex file. The resume.cls file provides the resume style used for structuring the
% document.
%
%%%%%%%%%%%%%%%%%%%%%%%%%%%%%%%%%%%%%%%%%

%----------------------------------------------------------------------------------------
%	PACKAGES AND OTHER DOCUMENT CONFIGURATIONS

%----------------------------------------------------------------------------------------

\documentclass[40]{resume} % Use the custom resume.cls style
\usepackage[left=0.30in,top=0.5in,right=0.30in,bottom=0.5in]{geometry} % Document margins
\newcommand{\tab}[1]{\hspace{.2667\textwidth}\rlap{#1}}
\newcommand{\itab}[1]{\hspace{0em}\rlap{#1}}

% [set line spacing
\renewcommand{\baselinestretch}{0.8} % normal line spacing
% [ url links
\usepackage{hyperref}
% ]

% [
\usepackage{etaremune} % reverse enumeration
% ]
\usepackage{enumitem}
\usepackage[parfill]{parskip}
\setlength{\parskip}{1.5pt} % line spacing for new lines
% \setlength{\parindent}{} % paragraph indentation
\name{Ojasv Kamal} % Your name
% \address{The University of Hong Kong, China}
\address{
% New York University \\ 
\href{mailto:kamalojasv2000@gmail.com}{kamalojasv2000@gmail.com} \\ \url{https://github.com/kamalojasv181} \\ (+91) 9215650602} % Line 3
% \address{Department of Computer Science} % Line 1

\begin{document}
% Examples to Learn from: https://yardi.people.si.umich.edu/Schoenebeck_CV.pdf
%----------------------------------------------------------------------------------------
%	EDUCATION SECTION
%----------------------------------------------------------------------------------------
\begin{rSection}{Education}
\vspace{0.3cm}

{\bf Indian Institute of Technology Kharagpur } \hfill { Kharagpur, India} 
\\ B.Tech M.Tech Dual Degree, Mechanical Engineering; CGPA: 8.85/10 \hfill { \em  2018 – Present}
\vspace{0.2cm}
\\
{\bf Sukriti World School } \hfill { New Delhi, India} 
\\ All India Senior School Certificate Examination (AISSCE); Percentage: 92.8/100 \hfill { \em  2017 - 2018}
\vspace{0.2cm}
\\
{\bf Scholars Rosary Senior Secondary School } \hfill { Rohtak, India} 
\\ All India Secondary School Examination (AISSE); CGPA: 10/10 \hfill { \em  2015 - 2016}

\end{rSection}


%----------------------------------------------------------------------------------------
%	PUBLICATIONS SECTION
%----------------------------------------------------------------------------------------

\begin{rSection}{Publications}

  \vspace{4pt}
  \begin{itemize}[leftmargin=0.15in]

    \vspace{4pt}
    \item \textbf{Moûsai: Text-to-Music Generation with Long-Context Latent Diffusion} \\
          Flavio Schneider$^*$, \textbf{Ojasv Kamal$^*$} (equal contribution), Zhijing Jin, Bernhard Schölkopf

          Available at
          \url{https://arxiv.org/abs/2301.11757}

          \vspace{4pt}

    \item \textbf{CLadder: A Benchmark to Assess Causal Reasoning Capabilities of Language Models}\\
          Zhijing Jin, Yuen Chen, Felix Leeb, Luigi Gresele, \textbf{Ojasv Kamal}, LYU Zhiheng, Kevin Blin, Fernando Gonzalez Adauto, Max Kleiman-Weiner, Mrinmaya Sachan, Bernhard Schölkopf.

          \emph{In Proceedings of the Advances in Neural Information Processing Systems 36} (NeurIPS 2023) \\Available at
          \url{https://openreview.net/forum?id=e2wtjx0Yqu}

    \item \textbf{When to Make Exceptions: Exploring Language Models as Accounts of Human Moral Judgment}
          Zhijing Jin, Sydney Levine, Fernando Gonzalez, \textbf{Ojasv Kamal}, Maarten Sap, Mrinmaya Sachan, Rada Mihalcea, Josh Tenenbaum, Bernhard Schölkopf.

          \emph{In Proceedings of the Advances in Neural Information Processing Systems 35} (NeurIPS 2022) \\Available at
          \url{https://arxiv.org/abs/2210.01478}

          \vspace{4pt}

    \item \textbf{Adversities are all you need: Classification of self-reported breast cancer posts on Twitter using Adversarial Fine-tuning} \\
          Adarsh Kumar$^*$, \textbf{Ojasv Kamal$^*$}, Susmita Mazumdar$^*$ (Equal Contribution)

          \emph{In Proceedings of the Sixth Social Media Mining for Health (SMM4H) Workshop and Shared Task NAACL 2021}. Available at \url{https://aclanthology.org/2021.smm4h-1.22/}
          \vspace{4pt}

    \item \textbf{Hostility Detection in Hindi leveraging Pre-Trained Language Models} \\
          \textbf{Ojasv Kamal$^*$}, Adarsh Kumar$^*$ (equal contribution), Tejas Vaidhya.

          \emph{International Workshop on Combating Online Hostile Posts in Regional Languages during Emergency Situation, AAAI 2021}. Available at \url{https://link.springer.com/chapter/10.1007/978-3-030-73696-5_20}
  \end{itemize}
\end{rSection}

\begin{rSection}{Research Experience}
  \vspace{0.3cm}
  {\bf Causal Relation Extraction from Human Conversations} \hfill {\em Jul 2022 - Present} 
\\ \textit{ETH Zürich, Language, Reasoning and Education Lab
, with Prof. Mrinmaya Sachan} \hfill { \em }
\begin{itemize}
\item Surveyed \textbf{30} datasets containing natural language text with labeled cause-effect span pairs.
\item Quantified the ability of BERT model to extract cause-effect spans in a supervised setting.
\item Compared this performance to zero-shot performance of GPT-3 by tuning over \textbf{250} prompts
\end{itemize}
\vspace{0.2cm}
{\bf Analysing Meta-Contrastive Learning for Few-Shot Slot Filling} \hfill {\em Dec 2021 - Present} 
\\ \textit{NUS, Web Information Retrieval / Natural Language Processing Group, with Prof. Min-Yen Kan} \hfill { \em }
\begin{itemize}
\item Interpreted the complex process of \textbf{meta-learning} by studying the two levels of base and meta-learning separately
\item Quantified the effect of using contrastive learning as the base learning algorithm for few-shot slot filling
\item Performed \textbf{42} ablation tests on \textbf{7} dataset domains to study the effect of removing each domain
\item Correlated the performance drop with \textbf{19} out-of-domain metrics representing text properties to interpret the role of contrastive learning in meta learning
\item Working on completing the post-result analysis to publish in a peer-reviewed conference
\end{itemize}

\vspace{0.2cm}
{\bf Conversational Recommender System Toolkit} \hfill {\em Jul 2022 - Nov 2022} 
\\ \textit{NUS, Web Information Retrieval / Natural Language Processing Group, with Prof. Min-Yen Kan} \hfill { \em }
\begin{itemize}
\item Built a user preference estimation module using factorization
machine to understand user preferences
\item Added a recommendation module to accomplish recommendations in a short number of turns using reinforcement learning
\item Created an end-to-end pipeline to incorporate the possibility of noise and uncertainty in language understanding
\end{itemize}

\vspace{0.2 cm}

{\bf When to Make Exceptions: Exploring Language Models as Accounts of Human Moral Judgment} \hfill { } 
\\ \textit{ETH Zürich, Language, Reasoning and Education Lab
, with Prof. Mrinmaya Sachan} \hfill {\em Mar 2022 - May 2022 }
\begin{itemize}
\item Investigated the potential of LLMs like GPT-3 and Delphi to reason in various moral scenarios by analyzing with \textbf{50} custom prompts
\item Scraped average prices of \textbf{220} services from \href{https://www.fiverr.com/?utm_source=306294&utm_medium=cx_affiliate&utm_campaign=&afp=63975b6bf036a80340a5ecd1&cxd_token=306294_17458287_|afp0:63975b6bf036a80340a5ecd1|afp1:1660_&show_join=true}{Fiver} to estimate these models' understanding of their value
\item Research was acclaimed as an oral presentation at Neural Information Processing System \textbf{(NeurIPS) 2022}
\end{itemize}
\pagebreak


{\bf Causal and Anticausal Prompting} \hfill {\em Mar 2022 - Jun 2022 } 
\\ \textit{ETH Zürich, Language, Reasoning and Education Lab
, with Prof. Mrinmaya Sachan} \hfill { \em }
\begin{itemize}
\item Designed causal and anticausal prompts using customer reviews from the Yelp dataset
\item Predicted customer ratings through these prompts using language models like RoBERTa and GPT-2
\item Employed \textbf{adversarial attack} on the model to establish superior robustness of causal prompts
\end{itemize}


\vspace{0.2cm}
{\bf Improving Automatic Speech Recognition (ASR) on Out-of-Vocabulary words} \hfill { \em May 2022 - Jul 2022} 
\\ \textit{Sprinklr, Machine Learning Team} \hfill { \em }
\begin{itemize}
\item Incorporated domain-specific data into the encoder and decoder of Sprinklr's ASR system.
\item Improved the decoder by combining the probabilities of the ASR model with a language model built using a domain-specific corpus.
\item Supplied the encoder with information from related ASR datasets using \textbf{continual learning}.
\item Reduced the word error rate by \textbf{118\%} using the language model and further by \textbf{42\%} using continual learning.
\item Implemented language model pruning to reduce memory utilization by \textbf{91\%} and latency by \textbf{31\%}
\end{itemize}
\vspace{2pt}
\end{rSection}



%	EXAMPLE SECTION
%----------------------------------------------------------------------------------------

\begin{rSection}{Research Projects}
\vspace{6pt}

{\bf Adversities are all you need: Classification of self-reported breast cancer posts on Twitter using Adversarial Fine-tuning} \hfill{ } 

\begin{itemize}
\item Classified \textbf{5000} tweets as positive/negative cases of self-report of breast cancer.
\item Enhanced the robustness towards adversarial examples by introducing a \textbf{gradient-based perturbation}.
\item Improved the average micro F1 score by \textbf{5\%} across various pre-trained language models.
\item Sixth Social Media Mining for Health (\textbf{SMM4H}) Workshop and Shared Task collocated with NAACL 2021
\end{itemize}
\vspace{0.2cm}
{\bf Hostility Detection in Hindi leveraging Pre-Trained Language Models
} \hfill { } 

\begin{itemize}
\item Designed an auxiliary model approach to detect hostility occurrence and type of hostility in Hindi tweets
\item Improved the micro F1 score by \textbf{13\%} over the baseline using the above approach
\item Paper accepted at the International Workshop on Combating Online Hostile Posts in Regional Languages during Emergency Situation (\textbf{CONSTRAINT}), collocated with AAAI 2021
\end{itemize}

\vspace{0.2cm}
{\bf Unsupervised Domain Adaptation} \hfill { } 
\\ \textit{IIT Kharagpur, Swarm Robotics Lab, with Prof. Somesh Kumar} \hfill { \em }
\begin{itemize}
\item Identified and studied the problem of covariate shift in the semantic segmentation of robots in the lab
\item Imparted domain invariant features from source to the target domain using shared parameters of ``Coupled Generative Adversarial Network."
\item Optimised the number of parameters in the network by implementing the paper ``Unsupervised Domain Adaptation by Backpropagation."
\end{itemize}
\end{rSection}

\begin{rSection}{Awards \& Achievement}
\vspace{0.2cm}
\begin{itemize}
  \setlength{\itemindent}{-1em}
  \item \textbf{Joint Entrance Exam Advanced:} Secured an All India Rank of 2135 (top \textbf{1} percentile) in the examination
  \item \textbf{Joint Entrance Exam Main:} Bagged an All India Rank of 3621 (top \textbf{0.3} percentile) in the examination
  \item \textbf{Kishore Vaigyanik Protsahan Yojana 2018:} Achieved an All India Rank of 1487 (top \textbf{3} percentile)
  \item \textbf{Table Tennis:} Accomplished \textbf{1$^{st}$} position in Table Tennis state championship (under-12 category) and managed a peak national rank of \textbf{36} in 2011
\end{itemize}

\end{rSection}


%----------------------------------------------------------------------------------------
%	TECHNICAL STRENGTHS SECTION
%----------------------------------------------------------------------------------------

\begin{rSection}{Mentorships \& Volunteer Activity}
\vspace{0.3cm}
\begin{itemize}
  \setlength{\itemindent}{-1em}
  \item \textbf{Research Team Mentor:} Mentored a group of four junior researchers in the software team at Swarm Robotics lab, IIT Kharagpur. Two of them are leading the team now.
  \item \textbf{Open Source Contribution:} Contributed various differential equation solvers in the form of \textbf{3111} lines of code to the OrdinaryDiffEq.jl package of Julia's scientific computing library SciML.
  \item \textbf{Teaching Assistant:} Assisted fourth-year undergraduate students in tutorial problems for the course "Systems and Controls"
  \item \textbf{Social Work Volunteer:} Volunteered in organizing a food donation and a dental checkup camp for Rotary Club, Rohtak.
\end{itemize} 


\end{rSection}


\begin{rSection}{Skills and Coursework}
\vspace{0.2cm}
\begin{tabular}{ @{} >{\bfseries}l @{\hspace{6ex}} l }
Languages &  English, Hindi, French, Sanskrit  \\
Programming Skills &  Python, C/C++, Julia, Javascript, MATLAB \\
Design Skills & Photoshop, Unity, WordPress, Prezi,
Microsoft Suite (Word, Excel, Powerpoint) \\
Technologies & Pytorch, TensorFlow, Keras \\
University Level Courses & Natural Language Processing, Probability and Statistics, Calculus, Linear Algebra\\
MOOCs & Machine Learning, Deep Learning, Reinforcement Learning, Computer Vision, \\ & Multitask and Meta Learning
% Interests  & Biking, Hiking, Yoga, Meditation, Archaeology and History \\

\end{tabular}

\end{rSection}



\end{document}